%%=========================================
\chapter[Conclusions]{Conclusions, Discussion, and Recommendations for Further Work}
\label{chp:conclusions}
\begin{info}
	In this final chapter you should sum up what you have done and which results you have got. You should also discuss your findings, and give recommendations for further work.
\end{info}
%%-----------------------------------------



%%=========================================
\section{Summary and Conclusions}
\label{sec:summary}
\begin{info}
	Here, you present a brief summary of your work and list the main results you have got. You should give comments to each of the objectives in Chapter 1 and state whether or not you have met the objective. If you have not met the objective, you should explain why (e.g., data not available, too difficult).

	This section is similar to the Summary and Conclusions in the beginning of your report, but more detailed---referring to the the various sections in the report.
\end{info}
%%-----------------------------------------



%%=========================================
\section{Discussion}
\label{sec:discussion}
\begin{info}
	Here, you may discuss your findings based on your results, their strengths and limitations. Note that this discussion is more high level than discussions made in relation to results you have achieved and presented in the previous chapter. The discussion here should put your work in larger context. You may address if you achieved what you had intended to do, why not (if you did not), if you got results in which you did not expect, why the results are important, why there are limitations in using the results, or if there are opportunities to transfer your results and findings into other domains, and so on.
\end{info}
%%-----------------------------------------



%%=========================================
\section{Recommendations for Further Work}
\label{sec:furtherwork}
\begin{info}
	You should give recommendations to possible extensions to your work. The recommendations should be as specific as possible, preferably with an objective and an indication of a possible approach.

	The recommendations may be classified as:
	\begin{itemize}
	\item Short-term
	\item Medium-term
	\item Long-term
	\end{itemize}
\end{info}
%%-----------------------------------------



%%=========================================